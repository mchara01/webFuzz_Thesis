\chapter{Introduction}
\pagenumbering{arabic} % start arabic page numbering
\minitoc
\vspace*{1cm}

In the introductory chapter to this thesis we analyse what motivated us to explore this research area of grey-box fuzzing and start this project. We also discuss related studies in the field of fuzzing, the contribution that this thesis makes to it while outlining the chapter topics to follow.

% NEW SECTION
\section{Motivation}
Fuzzing is now widely recognised as an essential process for discovering hidden bugs in computer software. Automated software testing or fuzzing is a tried and tested method of generating or mutating inputs and passing them to programs in search of bugs. The spark in the fuzzing 'revolution' to discover bugs in software in an automated process has been precipitated with the introduction of AFL ~\cite{zalewski2015american}, a state-of-the-art fuzzer that produces feedback during fuzzing by leveraging instrumentation of the analysed program. By creating this \textit{feedback loop}, fuzzers can significantly improve their performance as they can determine whether an input is interesting, namely it triggers a new code path, and uses that input to produce other test cases.

Software-testing plays a vital role in the software development cycle because when vulnerabilities are present, they can have severe and even irreparable consequences. By exploiting software bugs, adversaries can perform data breaches, install malicious malware or even take complete control of a device. Detecting bugs before they get exploited is a possible but demanding task. Mainly because bugs are triggered when an unexpected input is given to the program, something which is difficult to fully simulate through statically written unit tests. This is due to the fact unit tests usually revolve around expected inputs to test the intended functionality of code ~\cite{aschermann2019nautilus}.

Although automated software testing has become an attractive field of research, it still has a long way to go, especially for web applications ~\cite{doupe2010johnny}. As the Internet infrastructure expands, much more of the software written in native code(precompiled program in the CPU's machine language) is migrating to web applications. This process attracts many more malicious attacks on web applications. This predicates a strong need for the development of automated vulnerabilities scanners that target web applications.
 
% NEW SECTION REPHRASE
\section{Related Work}
Numerous fuzzers recently developed try to optimize the fuzzing process by proposing different methodologies ~\cite{godefroid2012sage, stephens2016driller, rawat2017vuzzer, aschermann2019nautilus, aschermann2019redqueen, hoffman2020Was, osterlund2020parmesan}. 
For example, most of the fuzzers take advantage of instrumentation at the binary or source level. This is done by inserting code in the program to receive feedback when a code block is
triggered so the fuzzer can adjust the generated inputs to improve code coverage. 
Other fuzzing methodologies utilize symbolic/concolic execution for extracting useful information about the program and use that information for improving the input generation process ~\cite{stephens2016driller,godefroid2005dart,godefroid2012sage}. However, all these fuzzers are currently targeted towards finding vulnerabilities in native code while web applications - which do not run on native code - have received limited attention. A more detailed analysis is given in Chapter ~\ref{sec:relatedwork}.

Traditionally, fuzzers come under three categories; black-, white- or grey-box which are clearly define for native applications. When it comes to web applications grey-box methods have not been defined, so our mission was to produce a prototype process inspired by work done on native applications, more precisely AFL.

\section{Contributions}
In this thesis, \pname{} is proposed. It is a prototype grey-box fuzzing tool for web applications. Today, the only fuzzers available for web applications are developed to behave in a black-box fashion ~\cite{doupe2010johnny}. That is to say they use brute force to bombard their targets with URLs that embed known web-attack payloads. Recently, there were breakthroughs with white-box fuzzing also ~\cite{navex2018,Borges2018BaZINGAWF}, that combine static analysis and concolic testing with fuzzing. Unlike the Black-/White-box fuzzing approach, \pname{} initially instruments the targeted web application by adding code that tracks all control flows triggered by an input and notifies the fuzzer, accordingly. 
Notifications can be embedded in the web application's HTTP response using custom headers or outputted to a shared log file or memory region. 
Subsequently, the fuzzer begins sending requests to the target and analyses the responses to detect any interesting requests that would later help to improve the code coverage and as a result, trigger vulnerabilities embedded deep in the web application's code.

The following contributions are made in this thesis:

\begin{enumerate}

\item We design, implement and evaluate \pname{}, the first grey-box fuzzer realized for discovering vulnerabilities in web applications. \pname{} applies instrumentation on the target web application for guiding the entire fuzzing process. Instrumentation can be applied on the AST level of PHP-based web applications for creating a feedback loop and utilizing it in order to increase code coverage.
\item We thoroughly evaluate \pname{} in terms of coverage, throughput and efficiency in finding unknown bugs. For a better understanding of the measured capabilities of \pname{} we compare our results with three existing web-application fuzzers. \pname{} is the only fuzzer that
reports coverage information. Specifically, \pname{} can cover about 21.5\% of WordPress, which has a codebase of approximately half a million lines of PHP code, in 50 hours of fuzzing. As expected, \pname{} is slower, in terms of throughput, due to the involved instrumentation. In fact, another popular fuzzer, Wfuzz ~\cite{wfuzz} is three times faster when fuzzing Drupal, but this is something to be expected, since the reduction of the throughput due to the instrumentation pays off in increased coverage in the long run. Finally, \pname{}, compared to the other three fuzzers, finds the most injected vulnerabilities (30 with the second one being Wfuzz with 28) during a fuzzing session that lasts 65 hours. The evaluation of \pname{} can be seen in detail in Chapter ~\ref{sec:evaluation}
\item To foster further research in the field, \pname{} will be released as open source.

\end{enumerate}

\section{Thesis Outline}
This thesis consists of eight chapters. In the first chapter we present our inspiration for undertaking this research, related work on the said topic and the contributions made in this thesis. In the second chapter we state any relevant background information required to grasp the perspective of this work. Continuing to the third chapter, the architecture of the tool is discussed on a higher level without delving into too much implementation details. The fourth chapter is dedicated to discussing the technical aspects of the fuzzing tool developed. The fifth chapter is an evaluation to see how well \pname{} performs in terms of finding bugs, code coverage and throughput against other fuzzers. In the sixth chapter, we review the limitations faced during the research process while unfurling what future plans we have for our tool. In the seventh chapter we elaborate on the related work made in the area of fuzzing over recent years. In the eighth and final chapter we summarise and reflect on the research done, and conclude on the evaluation of our approach.
