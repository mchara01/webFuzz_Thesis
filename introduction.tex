\chapter{Introduction}
\pagenumbering{arabic} % start arabic page numbering
\minitoc
\vspace*{1cm}

A highly automated testing technique that covers numerous boundary cases
using invalid data (from files, network protocols, API calls, and other targets)
as application input to better ensure the absence of exploitable vulnerabilities.
The name comes from modem applications’ tendency to fail due
to random input caused by line noise on fuzzy telephone lines.A highly automated testing technique that covers numerous boundary cases
using invalid data (from files, network protocols, API calls, and other targets)
as application input to better ensure the absence of exploitable vulnerabilities.
The name comes from modem applications’ tendency to fail due
to random input caused by line noise on fuzzy telephone lines.


Numerous fuzzers have been developed in the past few
years that try to optimize the fuzzing process by proposing
various methodologies [7, 8, 20, 21, 35, 42, 46]. For instance,
most of the fuzzers take advantage of instrumentation on
the source or binary level. That is, inserting code to the
program in order to receive feedback when a code block gets
triggered and try to adjust the generated inputs to improve
code coverage. Others utilize concolic/symbolic execution
in order to extract useful information about the program
and use that information for improving the input generation
process [20, 21, 46]. However, all these fuzzers are currently
targeted towards finding vulnerabilities in native code, while
web applications have received limited attention.


mutation-based fuzzer, might actively see the code paths executed in the target and make adjustments accordingly, which is very smart. EFS and AFL do exactly this

even google is very active in the industry of fuzzers https://github.com/google/oss-fuzz, atheris
% NEW SECTION
\section{Motivation}

% NEW SECTION
\section{Related Work}

% NEW SECTION
\section{Contributions}

\section{Outline Contents}