\chapter{Conclusion}
\label{sec:conclusion}
\vspace*{0.25cm}

Fuzzing has evolved significantly in analysing native applications, becoming a 'hot' field for research. While used extensively to uncover destructive bugs and security vulnerabilities in native apps, web applications have received scarce attention.

In this thesis, we presented \pname{}, a prototype open-source grey-box fuzzer for discovering Cross-Site Scripting vulnerabilities in web applications. \pname{} utilises \emph{instrumentation} on the target web application to produce a \emph{feedback loop}, employing it to boost code coverage score.
Consequently, it increased the total of potential vulnerabilities found.

For the evaluation of \pname{}, we used four web applications; namely WordPress, Firefly-III, Mautic and Drupal, for the following three metrics: competence in detecting Reflected Cross-Site Scripting bugs, Throughput and Global Code Coverage.

Regarding the first metric, \pname{} was able to detect the most artificially injected vulnerabilities compared to the other three black-box fuzzers in the test. More specifically, it was able to expunge \emph{30} bugs followed by Wfuzz with \emph{28}. 

Secondly, in terms of throughput, unfortunately, \pname{} does not match the throughput of native applications fuzzers such as AFL nor black-box web-app fuzzers such as Wfuzz as the overhead from the \emph{instrumentation} is hefty. 

Other outcomes in our third metric, suggest that our fuzzing tool can achieve coverage of the WordPress and Drupal codebase up to \emph{21.5\%} and \emph{27.3\%} respectively, in \emph{4.2} days of fuzzing.

Following a "borderline" rejection on the initial submission of this thesis at CODASPY 2021, the required modifications and clarifications were made, giving us confidence that a more comprehensive study will be accepted when resubmitted to DIMVA 2021 for publication.