\chapter{Future Work}
\label{sec:futurework}
\minitoc
\vspace*{1cm}


netmap

Plans to include more crucial web-app vulnerabilities in our detection suite that can have as much serious security implications as Cross-Site Scripting. Such vulnerabilities can be found at OWASP Top 10 ~\cite{owasp2017} and may include the likes of Injection and Broken Authentication.  Injection flaws, such as SQL and NoSQL occur when untrusted data is sent to an interpreter / database as part of a query. For this particular vulnerability various known payloads have already been collected ~\cite{seclist}, the same way as the XSS payloads, and are stored in the repository waiting for the respective functionality to be added to \pname{}.

The are plans to implement a better and more efficient string-matching algorithm that will decrease the number of false positives we currently record. This can be achieved by taking into consideration the location of the payload in the HTML document. This kind of improvement will enable us to detect Cross-Site Scripting vulnerabilities that get triggered due to HTML attributes such as {\tt onchange } and {\tt onclick}, and not due to the HTML's <script>.

Also, include more Python modules to improve the overall performance of \pname{}. Since our fuzzer requires a lot of file I/O in order to its logging work, the {\tt mmap } module can be utilised by using lower-level operating system APIs to load a file directly into computer memory and read/write files as if they were one large string or array ~\cite{mmap}. Another module that may boost the performance of \pname{} is aiomultiprocess ~\cite{aiomultiprocess}. As we briefly mentioned in Chapter ~\ref{sec:background}, AsyncIO is limited to the speed of GIL , and multiprocessing entails spreading tasks over a computer's cores. Combining the two we can overcome the obstacles and truly achieve parallelism in Python.

As Donald Knuth has said, "Premature optimization is the root of all evil (or at least most of it) in programming".