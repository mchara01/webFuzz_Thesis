\section*{\LARGE{Abstract}}

Testing software is a common practice for exposing unknown vulnerabilities
in security-critical programs that can be exploited with malicious intent.
A bug-hunting method that has proven to be very effective is a technique
called fuzzing. 

Specifically, this type of software testing is frequently in the
form of fuzzing of native code, which includes subjecting the program to
enormous amounts of unexpected or malformed inputs in an automated fashion.

This is done to get a view of their overall robustness to detect and fix
critical bugs or possible security loopholes. For instance, a program crash
when processing a given input may be a signal of memory-corruption
vulnerability.

Although fuzzing has significantly evolved in analysing native code, web
applications, invariably, have received limited attention until now. 

This thesis explores the technique of grey-box fuzzing of web applications and the
construction of a fuzzing tool that automates the process of discovering
bugs in web applications. We design, implement and evaluate webFuzz, which is a prototype grey-box
fuzzer for web applications. 

Our fuzzing tool was successful in leveraging instrumentation for detecting Reflective Cross-Site Scripting (XSS) vulnerabilities faster than other
black-box fuzzers. The functionality of webFuzz is demonstrated using the most popular open-source web applications written in PHP.